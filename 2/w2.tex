\documentclass[12pt,letterpaper]{article}

\usepackage{quiver}
\usepackage{Pack}
\usepackage{PackMath}
\usepackage[placement=bottom,angle=0,color=black!40,scale=3,hshift=88,vshift=5]{background}
\usepackage{enumitem}
\setlist{nosep}
\usepackage{hyperref}
\usepackage{cleveref}


\usepackage{minted}
\newcommand{\code}[1]{\texttt{#1}}
% \newenvironment{codeblock}{\VerbatimEnvironment \begin{minted}[linenos,breaklines]{python}}{\end{minted}}


\backgroundsetup{contents={-\thepage-}}
\makeatletter
\renewcommand*\env@matrix[1][*\c@MaxMatrixCols c]{%
   \hskip -\arraycolsep
   \let\@ifnextchar\new@ifnextchar
   \array{#1}}
\makeatother

\usepackage{fancyhdr}
\pagestyle{fancy}
\fancyhf{}
\fancyhead[LE,RO]{Workshop 2}
\fancyhead[CE,CO]{Yuxi (Jaden) Long}
\fancyhead[RE,LO]{Duke BioByte}
%\fancyfoot[CE,CO]{}

% For progress tracker
\newcommand{\ns}{{\color{red} Not Started}}
\newcommand{\ip}{{\color{orange} In Progress (Stuck)}}
\newcommand{\td}{{\color{blue} In Progress (To-Do)}}
\newcommand{\fin}{{\color{green} Finished}}
\newcommand{\qm}{{\color{violet} Finshed, But In Doubt}}
\newcommand{\contradiction}{\Rightarrow\!\Leftarrow}
\renewcommand{\im}{\mathrm{im\,}}
\renewcommand{\tilde}{\widetilde}
\setlength{\parindent}{0em}
\setlength{\parskip}{0.5em}


\usepackage{notomath}
\usepackage{multicol}
\newcommand{\Hint}{\textcolor{violet}{\textit{Hint: }}}
\newcommand{\Solution}{\textcolor{MidnightBlue}{\textbf{Solution: }}}


\title{\textbf{Sequence Alignment and Variant Calling}}
\author{\textit{Computational Tools for the Working Biologist. Workshop 2}}
\date{Yuxi (Jaden) Long}

\begin{document}
\maketitle
\thispagestyle{empty}

\vspace{1em}

\noindent
Short-read DNA or RNA sequencing is cheap and fast, and generates a lot of data. As a result, genomics is the most relevant and fundamental computational task that a biologist would encounter.

\noindent
\textbf{Central idea/technique:} Understand the information in genomic sequencing raw data, and learn tools to manipulate them.

\textbf{Practice:} Perform variant calling on a sequencing datum from beginning to end.

\begin{itemize}
   \item Quality control: fastp
   \item Reference data: NCBI GenBank, NCBI SRA
   \item Reference based sequence alignment: BWA
   \item Viewing your alignment: IGV
   \item Alignment processing: samtools, picard
   \item Variant calling: GATK HaplotypeCaller
\end{itemize}

\textit{This workflow is largely adapted from the independent study of Hector de Galard in Fall 2019.}

\section{Short read sequencing}

In principle, genome sequencing can be considered as measuring the nucleotide sequences of fragments that come the genome of interest. Each such measurement is called a \textit{read}. Naturally, each read is associated with a \textit{read length}, which is the length of each fragment sequenced, measured in base pairs.

\begin{table}[]
\begin{center}
\begin{tabular}{|l|l|l|}
\hline
Sequencing technologies & Short-read & Long-read   \\
 &  &    \\
 &  &    \\
 &  &
\end{tabular}
\end{center}
\end{table}


% With our current technology, it is much cheaper to perform \textit{short-read sequencing} than \textit{long-read sequencing}.

% \textit{Short-read sequencing} is dominated by Illumina machines. The technology is amazing, but to a bioinformatics standpoint, what matters

\end{document}
